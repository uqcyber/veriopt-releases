\documentclass[11pt,a4paper]{article}
\usepackage{isabelle,isabellesym}
\usepackage{amssymb}
\usepackage{tikz-cd}

% this should be the last package used
\usepackage{pdfsetup}

% urls in roman style, theory text in math-similar italics
\urlstyle{rm}
\isabellestyle{it}

% for uniform font size
%\renewcommand{\isastyle}{\isastyleminor}

% display anything in a snip without any markup
\newcommand{\Snip}[1]{\isanewline\isanewline}
\newcommand{\EndSnip}{\isanewline\isanewline}
\newcommand{\induct}[1]{}


\begin{document}

\title{GraalVM Canonicalization Optimizations}
\maketitle

\begin{abstract}
This document presents the canonicalization rules
which are present in the GraalVM compiler.
First, individual rules are encoded in a high-level
domain specific language.
As these optimizations are encoded, a proof of semantics
preservation is given.
Next, rules are combined via a tactic language.
The combined rules are then proved to be terminating.
Finally, optimization phases are composed of the combined rules
and generated into Java code.
\end{abstract}

\pagebreak

\tableofcontents

\pagebreak

% sane default for proof documents
\parindent 0pt\parskip 0.5ex

% generated text of all theories
\input{CanonicalizationSyntax}

% optional bibliography
%\bibliographystyle{abbrv}
%\bibliography{root}

\end{document}

%%% Local Variables:
%%% mode: latex
%%% TeX-master: t
%%% End:
